\documentclass{article}
\usepackage[margin=.9in]{geometry}              
\geometry{letterpaper}
\usepackage[parfill]{parskip}               
\usepackage{amssymb,amsmath}
\usepackage{amsthm}
\usepackage{mathtools}
\usepackage{enumerate}
\usepackage{gensymb}
\usepackage{tkz-graph}
\usepackage{tikz}
\usetikzlibrary{matrix}
\usepackage{cancel}
\usepackage[final]{pdfpages}
\usepackage{tabularx}


\usepackage[
backend=bibtex,
style=numeric,
citestyle=numeric,
maxbibnames=5,
sorting=none]{biblatex}

\bibliography{4f_pred} 

\usepackage{hyperref}
\hypersetup{
	colorlinks=true,
	linkcolor=blue,
	filecolor=magenta,      
	urlcolor=blue	,
}

\urlstyle{same}

\makeatletter
\def\blx@maxline{77}
\makeatother


\graphicspath{{images/}}

\newenvironment{mycenter}[1][\topsep]
  {\setlength{\topsep}{#1}\par\kern\topsep\centering}% \begin{mycenter}[<len>]
  {\par\kern\topsep}% \end{mycenter}

\newenvironment{problem}{\rightskip1in}{\begin{mycenter}[-2pt]\rule{1.0\textwidth}{.4pt}\end{mycenter}}
 
\relpenalty=9999
\binoppenalty=9999

\theoremstyle{remark}
\newtheorem*{claim}{\\Claim}
\newtheorem*{lemma}{\\Lemma}
\newcommand{\inv}{^{-1}}
\newcommand{\pd}[2]{\frac{\partial #1}{\partial#2}}
\newcommand{\td}[2]{\frac{d#1}{d#2}}
\newcommand{\tab}{\hspace*{2em}}
\newcommand{\tn}[2]{\tensor{#1}{#2}}
\newcommand{\bp}[1]{\left(#1\right)}
\renewcommand{\t}[1]{\text{#1}}
\newcommand{\mb}[1]{\mathbb{#1}}
\newcommand{\mds}[1]{\mathds{#1}}
\newcommand{\mc}[1]{\mathcal{#1}}
\newcommand{\comp}[1]{\overline{#1}}
\newcommand{\AI}{A^{(4)}_{0|<I_{\t{in}}>}}
\newcommand{\A}[1]{A^{(#1)}}
\newcommand{\lo}{\lambda_\t{opt}^{(4)}}
\newcommand{\ip}{$I\rightarrow P$ }

\newcolumntype{Y}{>{\centering\arraybackslash}X}

\linespread{1.25}
\renewcommand{\vec}[1]{\boldsymbol{#1}}
\newcommand{\horrule}[1]{\rule{\linewidth}{#1}}
\newcommand{\abs}[1]{\left|#1\right|}%
\DeclarePairedDelimiter\ceil{\lceil}{\rceil}
\DeclarePairedDelimiter\floor{\lfloor}{\rfloor}

\title{HWPSS Runthrough}
\author{Jack Lashner}
\date{}




\begin{document}
\maketitle

This document will detail all major steps in the HWPSS calculation for the small aperture telescope where the band center is at 145 GHz, and the FOV is 30$\degree$.

\section{IP Calculation}
The IP is dominated by three elements, the two aluminum filters on the sky side of the HWP, and the window.
Transmission and Reflection coefficients for $s$ and $p$ polarization are calculated separately using the Transfer Matrix Method (using the python tmm package), and averaged across the bandwidth of the detector.
The absorption coefficient is then determined by:
\[1 = T_s + R_s + A_s.\]
where all coefficients are between 0 and 1.

The indices of refraction used were calculated using the dielectric coefficients and loss tangents given here: \url{http://www.eccosorb.com/Collateral/Documents/English-US/dielectric-chart.pdf}, and are $n = 1.5 + i .0001$ for glass and $n=3.1 +i .00008$ for the filter.

Two layers of AR coating were used to optimize for both 95 GHz and 145 GHz. 
The outer and inner layers had indices of refraction $n = n_0^{1/3}$ and $n_0^{2/3}$ respectively.



The Instrumental Polarization of each element is the differential transmission:
\[T_{s - p} = \frac{T_s - T_p}{2}\]
The factor of two is necessary because both $T_s$ and $T_p$ vary between 0 and 1, 
while we requre the incident power to be normalized to 1.

Similarly, the polarized emissivity is calculated using the differential absorption:
\[A_{s - p} = \frac{A_s - A_p}{2} .\]
All coefficients for the window are given in Table \ref{tab:opt_coeffs}.


\begin{table}
\centering
\begin{tabular}{|c|c|c|c|}
\hline
&\multicolumn{3}{|c|}{Window}\\
\hline
 & Trans (\%) & Refl (\%) & Abs (\%) \\
 \hline
S    & 99.06 &0.53 & 0.409\\
P    & 99.15 & 0.43  & 0.409\\
S-P & .049  & .049& $1.2\times 10^{-5}$\\
 \hline
\end{tabular}
\begin{tabular}{|c|c|c|c|}
	\hline
	&\multicolumn{3}{|c|}{Aluminum Filter}\\
	\hline
	& Trans (\%) & Refl (\%) & Abs (\%) \\
	\hline
	S    & 96.199 & 3.630 & 0.169\\
	P    & 96.685 & 3.144  & 0.169\\
	S-P & 0.242  & 0.242 & $4.06\times 10^{-5}$\\
	\hline
\end{tabular}
\caption{
	Band-averaged transmission, reflection, and absorption coefficients for both S and P types of polarization for a window with an incident angle of $15 \degree$.
	The S-P row is the differential coefficient, including the factor of 1/2.
.}
\label{tab:opt_coeffs}
\end{table}


\section{Propagation}
To calculate the polarized power on the detector, we need to calculate both the differential transmission and the emitted polarized power.

To calculate the instrumental polarization, we multiply the incident unpolarized power by the differential transmission $T_{s-p}$.
To calculate the polarized emission, the blackbody spectrum at the temperature of the optical element with emissivity $A_{s-p}$ is integrated across the detector bandwidth.
The total polarized power at the detector is the sum of these two
multiplied by the polarized cumulative efficiency of everything on the detector side of the element.
For each element, these values are given in Table \ref{tab:opt_power}.

This gives a total polarized power at the detector of around 0.0386 pW.


\begin{table}
	\centering
	\begin{tabular}{| c | c | c | c | c |}
		\hline
		                      			& Window (pW) & Filter 1 (pW) & Filter 2 (pW) & Total (pW)\\ \hline
		UnPol Inc Power on Element          & 10.52 &    19.08      &   18.86       & \\
		Pol Power from $T_{s-p}$ after Element  	 	&  0.00495 &   0.0454       &    0.0449      & \\
		Emitted Pol power after Element    &  $-3.8\times10^{-5} $ &    $-3.5 \times 10^{-4} $  & $-1.7 \times 10^{-4}   $    & \\
		Total Pol Power after Element  	     &  0.00495  &  0.0454   &  0.0449       & \\
		Cum. Pol. Efficiency & 0.389 & 0.400 & 0.413 & \\
		Pol Power at detector on Detector& 0.00192  &  0.01816   & 0.01854& 0.0386 \\
		\hline
	\end{tabular}
\caption{Polarized power (pW) generated by each major optical element. }
\label{tab:opt_power}
\end{table}

\section{Conversion to $K_\text{cmb}$}

To calculate  the conversion factor to $K_\text{cmb}$ in the sky, we follow Charlie Hill's sensitivity code and use the unpolarized power from the CMB which reaches the detector.
The detector bandwidth runs from $\nu_{\t{low}} =  125$ GHz to $\nu_{\t{high}}=165$ GHz, so we have

\[P_\text{cmb} = 
\frac{\eta_\t{tel}}{2} \int_{\nu_{\t{low}}}^{\nu_{\t{high}}} 
\frac{2 h \nu}{e^{h \nu/k T_\t{cmb}} - 1} d \nu
\]
where $\eta_\t{tel} = 0.3638$ is the total efficiency of the telescope, and the 1/2 is included because only half of the unpolarized power couples to the detector.
The conversion between pW is then 
\[
\td{P_\t{cmb}}{T_\t{cmb}} =
\frac{\eta_\t{tel}}{2} \int_{\nu_{\t{low}}}^{\nu_{\t{high}}} 
\frac{2 h^2 \nu^2}{ k T_\t{cmb}^2 } 
\frac{e^{h \nu/k T_\t{cmb}}}{\left(e^{h \nu/k T_\t{cmb}} - 1 \right)^2} d \nu 
= 0.1199 \frac{\t{pW}}{K_\t{cmb}}
\]

The total power in is then about 321 mK$_\t{cmb}$.

To get the Rayleigh Jean's temperature we do use the same method but with the Rayleigh Jean's approximation. 
\[P_\text{RJ} = 
\frac{\eta_\t{tel}}{2} \int_{\nu_{\t{low}}}^{\nu_{\t{high}}} 
2 k T_\t{cmb} d \nu = 
\eta_\t{tel} k T_\t{RJ} (\nu_{\t{high}} - \nu_{\t{low}})
\]
The conversion is then 
\[
\td{P_\t{RJ}}{T_\t{RJ}} = \eta_\t{tel} k  (\nu_{\t{high}} - \nu_{\t{low}}) = 0.2008\; \frac{\t{pW}}{K_\t{RJ}}
\]
This gives us a total power of 192 mK$_\t{RJ}$ .

\end{document}

