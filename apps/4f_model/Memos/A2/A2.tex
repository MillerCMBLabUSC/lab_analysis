\documentclass{article}
\usepackage[margin=.9in]{geometry}              
\geometry{letterpaper}
\usepackage[parfill]{parskip}               
\usepackage{amssymb,amsmath}
\usepackage{amsthm}
\usepackage{mathtools}
\usepackage{enumerate}
\usepackage{gensymb}
\usepackage{tkz-graph}
\usepackage{tikz}
\usetikzlibrary{matrix}
\usepackage{cancel}
\usepackage[final]{pdfpages}
\usepackage{tabularx}


\usepackage[
backend=bibtex,
style=numeric,
citestyle=numeric,
maxbibnames=5,
sorting=none]{biblatex}

\bibliography{A2} 

\makeatletter
\def\blx@maxline{77}
\makeatother


\graphicspath{{images/}}

\newenvironment{mycenter}[1][\topsep]
  {\setlength{\topsep}{#1}\par\kern\topsep\centering}% \begin{mycenter}[<len>]
  {\par\kern\topsep}% \end{mycenter}

\newenvironment{problem}{\rightskip1in}{\begin{mycenter}[-2pt]\rule{1.0\textwidth}{.4pt}\end{mycenter}}
 
\relpenalty=9999
\binoppenalty=9999

\theoremstyle{remark}
\newtheorem*{claim}{\\Claim}
\newtheorem*{lemma}{\\Lemma}
\newcommand{\inv}{^{-1}}
\newcommand{\pd}[2]{\frac{\partial #1}{\partial#2}}
\newcommand{\td}[2]{\frac{d#1}{d#2}}
\newcommand{\tab}{\hspace*{2em}}
\newcommand{\tn}[2]{\tensor{#1}{#2}}
\newcommand{\bp}[1]{\left(#1\right)}
\renewcommand{\t}[1]{\text{#1}}
\newcommand{\mb}[1]{\mathbb{#1}}
\newcommand{\mds}[1]{\mathds{#1}}
\newcommand{\mc}[1]{\mathcal{#1}}
\newcommand{\comp}[1]{\overline{#1}}
\newcommand{\AI}{A^{(4)}_{0|<I_{\t{in}}>}}
\newcommand{\A}[1]{A^{(#1)}}
\newcommand{\lo}{\lambda_\t{opt}^{(4)}}
\newcommand{\ip}{$I\rightarrow P$ }

\newcolumntype{Y}{>{\centering\arraybackslash}X}

\linespread{1.25}
\renewcommand{\vec}[1]{\boldsymbol{#1}}
\newcommand{\horrule}[1]{\rule{\linewidth}{#1}}
\newcommand{\abs}[1]{\left|#1\right|}%
\DeclarePairedDelimiter\ceil{\lceil}{\rceil}
\DeclarePairedDelimiter\floor{\lfloor}{\rfloor}

\title{A2 calculation}
\author{}

\begin{document}
\maketitle

In this document I will be recording the HWP emission models used, and will present data on the HWPSS 2f signal for a warm and cold HWP.

\section{Absorption Model}
[CITE PAPER]

When an absorption coefficient is introduced into the HWP, the transmitted electric fields become:
\[
{{E_t}^e}' = e^{\alpha_e d} E_{t}^{e}  \qquad \t{and} \qquad{{E_t}^o}' = e^{\alpha_o d} E_t^o
\]
The amount of power absorbed is then:
\[A^e(\nu) = 1 - \abs{{{E_t}^e}'}^2 - \abs{{{E_r}^e}'} \qquad \t{and} \qquad 
A^o(\nu) = 1 - \abs{{{E_t}^o}'}^2 - \abs{{{E_r}^o}}
\]
giving us a differential absorption
\[
A^{e-o} = \frac{ \abs{{{E_t}^o}'}^2 -  \abs{{{E_t}^e}'}^2}{2}
\sim
\frac{1}{2}\left[(1 - e^{\alpha_e d})^2 - (1 - e^{\alpha_o d})^2\right] E_t^2
\]

The coefficients $\alpha_e$ and $\alpha_o$ are experimentally determined, and are fitted by the polynomials
\[
\alpha_e = 1.47\times 10^{-7} \nu^{2.2}
\qquad
\alpha_o = 8.7\times 10^{-5} \nu + 3.1 \times 10^{-7} \nu^2 + 3.0 \times 10^{-10} \nu^3
\]
At 145 GHz, we have $\alpha_e = .00836$ and $\alpha_0 = .02004$.
Plugging these in gives a polarized emissivity of $\epsilon_\t{pol} = 1.503 \times 10^{-5}$.

\section{Results}




\printbibliography

\end{document}

