\documentclass{article}
\usepackage[margin=.9in]{geometry}              
\geometry{letterpaper}
\usepackage[parfill]{parskip}               
\usepackage{amssymb,amsmath}
\usepackage{amsthm}
\usepackage{mathtools}
\usepackage{enumerate}
\usepackage{gensymb}
\usepackage{tkz-graph}
\usepackage{tikz}
\usetikzlibrary{matrix}
\usepackage{cancel}
\usepackage[final]{pdfpages}


\usepackage[
backend=bibtex,
style=numeric,
citestyle=numeric,
maxbibnames=5,
sorting=none]{biblatex}

\bibliography{4f_pred} 

\makeatletter
\def\blx@maxline{77}
\makeatother


\graphicspath{{images/}}

\newenvironment{mycenter}[1][\topsep]
  {\setlength{\topsep}{#1}\par\kern\topsep\centering}% \begin{mycenter}[<len>]
  {\par\kern\topsep}% \end{mycenter}

\newenvironment{problem}{\rightskip1in}{\begin{mycenter}[-2pt]\rule{1.0\textwidth}{.4pt}\end{mycenter}}
 
\relpenalty=9999
\binoppenalty=9999

\theoremstyle{remark}
\newtheorem*{claim}{\\Claim}
\newtheorem*{lemma}{\\Lemma}
\newcommand{\inv}{^{-1}}
\newcommand{\pd}[2]{\frac{\partial #1}{\partial#2}}
\newcommand{\td}[2]{\frac{d#1}{d#2}}
\newcommand{\tab}{\hspace*{2em}}
\newcommand{\tn}[2]{\tensor{#1}{#2}}
\newcommand{\bp}[1]{\left(#1\right)}
\renewcommand{\t}[1]{\text{#1}}
\newcommand{\mb}[1]{\mathbb{#1}}
\newcommand{\mds}[1]{\mathds{#1}}
\newcommand{\mc}[1]{\mathcal{#1}}
\newcommand{\comp}[1]{\overline{#1}}
\newcommand{\AI}{A^{(4)}_{0|<I_{\t{in}}>}}
\newcommand{\A}[1]{A^{(#1)}}
\newcommand{\lo}{\lambda_\t{opt}^{(4)}}
\newcommand{\ip}{$I\rightarrow P$ }

\linespread{1.25}
\renewcommand{\vec}[1]{\boldsymbol{#1}}
\newcommand{\horrule}[1]{\rule{\linewidth}{#1}}
\newcommand{\abs}[1]{\left|#1\right|}%
\DeclarePairedDelimiter\ceil{\lceil}{\rceil}
\DeclarePairedDelimiter\floor{\lfloor}{\rfloor}

\title{Small Aperture HWPSS}
\author{}
\date{}




\begin{document}
\maketitle

\section{Intro}
Here is a brief description on the small aperture HWPSS calculation. 
The majority of the work is identical to the calculation of the large aperture HWPSS, with the exception of the IP coefficients.

\section{Calculating IP}
Unlike the large aperture system, rays hitting a given pixel enter the optics parallel to one another. 
Because of this a simple plane wave analysis should be reasonably accurate to calculate the IP of each element. 
Grant wrote a python script which feeds in the index of refraction and thickness of each optical element in the optical chain into tmm to calculate the differential transmission of the entire system.
Because my script requires the IP of each element individually, I separately ran the script for each element (and its AR coating) in isolation.
I am not sure if this is correct, since the IPs of each individual element do not sum up to the IP of the whole system (its off by something like 30\%).

I ran this for incident angles $\theta = 7.5\degree, 10 \degree, 12.5\degree, $ and $15\degree$,
corresponding to FOV values of $2 \theta$.
Because of the large incident angles, larger FOV's have more IP and thus a larger $\A4$ value.

The major source are the Aluminum filters, of which there are two skyside of the HWP. 
The window also contributes, but the IP coefficient is an order of magnitude smaller than the Aluminum filters.
The coefficients for these two optical elements at various incident angles are given in Table \ref{table:IP_coeffs}

\begin{table}[h]
\centering

\begin{tabular}{|c|c|c|}
\hline
$\theta $       & Aluminum Filter IP   & Window IP            \\
\hline
7.5$\degree$  & $1.9\times 10^{-5}$  & $7.09\times 10^{-7}$ \\
10$\degree$   & $5.85\times 10^{-5}$ & $3.8\times 10^{-6}$  \\
12.5$\degree$ & $1.38\times 10^{-4}$ & $1.4\times 10^{-5}$   \\
15$\degree$   & $2.74\times 10^{-4}$ & $3.88\times 10^{-5}$ \\\hline
\end{tabular}
\caption{ IP coefficients for various incident angles
}
\label{table:IP_coeffs}
\end{table}
\section{Optical Chain}
The optical chain and channel input files I am using are given in Grant's emails


\section{Results}
$\A4$ for a detector at the edge of the array for various FOV's are given below.
I am only giving results in $K_\text{CMB}$ because the HWPSS in pW is proportional to the efficiency of the aperture,
which depends on the F-Number.
Because you can have different F-numbers for the same FOV, this would add another dimension to the input.
The aperture efficiency cancels out when converting to $K_\text{CMB}$, so we only get one value for each FOV.
I can include different F-Numbers in the future if people are interested.

\begin{table}[h]
\centering

\begin{tabular}{|c|c|c|}
\hline
$\theta$ & \multicolumn{2}{|c|}{$\A4 $ (mK$_\text{CMB}$)} \\
\hline
      & Aluminum Filter IP   & Window IP            \\
\hline
15$\degree$  & 4.52  & 5.88 \\
20$\degree$  & 13.9 & 18.1  \\
25$\degree$  & 43.6 & 57.1   \\
30$\degree$  & 95 & 124.1 \\\hline
\end{tabular}
\caption{ $\A4$ for various incident angles
}
\label{table:HWPSS}
\end{table}

\end{document}

